\documentclass[a4paper,12pt]{article}
\usepackage[utf8x]{inputenc}
\usepackage{amssymb, amsmath}
\newcommand{\ubar}{\underbar}
\usepackage{fullpage}

\begin{document}
\begin{titlepage}
\title{Dispensa per il secondo compitino di Analisi II}
\author{Riccardo Iaconelli}
% Remove command to get current date 
\maketitle
\end{titlepage}

\begin{titlepage}
\tableofcontents
\end{titlepage}

\section{Successioni e serie}

Sia $\{f_n\}$ una successione di funzioni a valori reali o complessi su $E$.
$\{f_n\}$ converge \textbf{puntualmente} se: $$\forall \varepsilon>0, \forall x \in E\ \ \exists N=N(\varepsilon, x):\ \forall n \geq N\ \forall x\in E\ \ |f_n(x) - f(x)| < \varepsilon$$ ove $$f(x):=\lim_{n\to+\infty} f_n(x)$$
\subsection{Criteri di convergenza}

\paragraph{Successioni}
$\{f_n\}$ converge uniformemente se e solo se $\forall \varepsilon>0\ \ \exists N=N(\varepsilon):\ \forall n,m \geq N\ \forall x\in E$ si ha che $|f_n(x) - f_m(x)| < \varepsilon$.

\paragraph{Serie}
Sia $\{M_n\}$ una successione (non negativa?) tale che, (definitivamente?), $\forall x$ $|f_n(x)|\leq M_n$. Allora se $\sum M_n$ converge $\Rightarrow\sum f_n$ converge uniformemente.



\section{Equazioni differenziali}

\subsection{Equazioni particolari}
\subsubsection{Equazioni lineari}
Con un'equazione di forma:
$$y'(x) = P(x)y+Q(x)$$
Allora
$$y(x)=e^{- \displaystyle\int P(x)dx}\left(k + \int Qe^{-\displaystyle\int P(x)dx}dx\right)$$
\subsubsection{Equazione di Bernoulli}
Con un'equazione di forma:
$$y'(x) = P(x)y+Q(x)$$
con $\alpha \neq 0, 1$. Allora, si impone:
$$ z(x)=y(x)^{1-\alpha} $$
si risolve l'equazione lineare in $z$: $$z'(x)=(1-\alpha)P(x)z(x)+Q(x)$$ e, trovata $z(x)$ con la formula sopra, si ricava $y(x)$.

\subsubsection{Equazione di Riccati}
$$y'(x) = P(x)y^2+Q(x)y+R(x)$$
Conoscendo una soluzione particolare $\Psi(x)$, allora applico la sostituzione
$$y(x)=\Psi(x)+\dfrac{1}{z(x)}$$
e con un po' di passaggi si dimostra che alla fine
$$z'(x)= -\left[2\Psi(x)P(x)+Q(x)\right]z(x)-P(x)$$
che è lineare.
\subsection{Variabili separabili}
Se invece
$$y'(x) = f(x)g(y)$$
Allora ovviamente si risolve ricordando che
$$\int\dfrac{dy}{g(y)}=\int f(x)dx + c$$
\subsection{Equazioni omogenee}
$$y'(x) = f\left(\dfrac{y(x)}{x}\right)$$
Allora si sostituisce
$$z=\dfrac{y}{x}$$
da cui $$z'(x) = \dfrac{dz}{dx} = \dfrac{f(z)-z}{x}$$
e dunque
$$\int\dfrac{dz}{f(z)-z}=\int \dfrac{dx}{x} + c$$
sicché
$$\int\dfrac{dz}{f(z)-z} = \log|x| + c$$
\subsubsection{Equazioni lineari di ordine $n$}
...


\end{document}

