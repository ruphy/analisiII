\documentclass[a4paper,10pt]{book}
\usepackage[utf8x]{inputenc}
\usepackage{amssymb, amsmath}

\author{Studenti vari}
\title{Schema per il primo compitino}

\begin{document}

\maketitle

\section{Continuità e differenziabilità}
\subsection{Definizioni principali}

\subsection{Come risolvere un esercizio}

\subsection{Ricorda}
\subsection{Trucchetti}

\section{Massimi e minimi}
\subsection{Definizioni principali}

\paragraph{Hessiana}
Si definisce matrice Hessiana della funzione f:
$$  H_{f}(\mathbf{x})_{ij} = \frac{\partial^2 f(\mathbf{x})}{\partial x_i\, \partial x_j} $$
$$ H(f) = \begin{bmatrix} \frac{\partial^2 f}{\partial x_1^2} & \frac{\partial^2 f}{\partial x_1\,\partial x_2} & \cdots & \frac{\partial^2 f}{\partial x_1\,\partial x_n} \\ \\ \frac{\partial^2 f}{\partial x_2\,\partial x_1} & \frac{\partial^2 f}{\partial x_2^2} & \cdots & \frac{\partial^2 f}{\partial x_2\,\partial x_n} \\ \\ \vdots & \vdots & \ddots & \vdots \\ \\ \frac{\partial^2 f}{\partial x_n\,\partial x_1} & \frac{\partial^2 f}{\partial x_n\,\partial x_2} & \cdots & \frac{\partial^2 f}{\partial x_n^2} \end{bmatrix}$$
(Nota: se tutte le derivate seconde di f sono continue in una regione $\Omega$, allora l'hessiana di f è una matrice simmetrica in ogni punto di $\Omega$. (le derivate miste sono uguali) )


\subsection{Come risolvere un esercizio}
\begin{itemize}
\item Calcola le derivate parziali $\frac{\partial f}{\partial x}$, $\frac{\partial f}{\partial y}$, e trovane gli zeri (all'interno di D).
\item Usa la matrice Hessiana per determinare il tipo di punto stazionario.
\begin{itemize}
 \item Calcola gli autovalori della matrice [$A = H(f(\vec{x_0}))$; $det(A-\lambda I)$]
 \item Se sono tutti positivi, il punto è un minimo.
 \item Se sono tutti negativi, il punto è un massimo.
 \item Se sono tutti nulli, il punto è di sella.
 \item Altrimenti, il test non è conclusivo.
 \item FIXME: COME SI FA?
\end{itemize}
\item Possono esistere punti di minimo o massimo sul bordo:
\begin{itemize}
 \item Restringi la funzione al bordo.
 \item Calcola la derivata e valuta i punti stazionari (nota: se la funzione era da $\mathbb{R}^2$ a $\mathbb{R}$, ora sarà semplicemente una funzione reale).
 \item In caso la derivata non sia banale, approssimare il valore del punto stazionario, dando almeno un intervallo entro il quale è compreso.
\end{itemize}
\end{itemize}

\subsection{Ricorda}
\subsection{Trucchetti}

\section{Funzioni implicite}
\subsection{Definizioni principali}

\subsection{Come risolvere un esercizio}

\subsection{Ricorda}
\subsection{Trucchetti}
\section{Integrali}
\subsection{Definizioni principali}

\subsection{Come risolvere un esercizio}

\subsection{Ricorda}
\subsection{Trucchetti}
\section{Altro}
\subsection{Formulette}
Trigonometria - formule parametriche:
$$sin(\alpha) = \frac{2t}{1+t^2}$$
$$cos(\alpha) = \frac{1-t^2}{1+t^2}$$
$$tan(\alpha) = \frac{2t}{1-t^2}$$
ove $$t = tan(\frac{1}{2}\alpha)$$
\subsection{Nozioni utili}

\end{document}


