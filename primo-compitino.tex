\documentclass[a4paper,10pt]{book}
\usepackage[utf8x]{inputenc}
\usepackage{amssymb, amsmath}

\author{Studenti vari}
\title{Schema per il primo compitino}

\begin{document}

\maketitle

\section{Continuità e differenziabilità}
\subsection{Definizioni principali}

\subsection{Come risolvere un esercizio}

\subsection{Ricorda}
\subsection{Trucchetti}

\section{Massimi e minimi}
\subsection{Definizioni principali}
Si definisce matrice Hessiana della funzione f:
$$  H_{f}(\mathbf{x})_{ij} = \frac{\partial^2 f(\mathbf{x})}{\partial x_i\, \partial x_j} $$
$$ H(f) = \begin{bmatrix} \frac{\partial^2 f}{\partial x_1^2} & \frac{\partial^2 f}{\partial x_1\,\partial x_2} & \cdots & \frac{\partial^2 f}{\partial x_1\,\partial x_n} \\ \\ \frac{\partial^2 f}{\partial x_2\,\partial x_1} & \frac{\partial^2 f}{\partial x_2^2} & \cdots & \frac{\partial^2 f}{\partial x_2\,\partial x_n} \\ \\ \vdots & \vdots & \ddots & \vdots \\ \\ \frac{\partial^2 f}{\partial x_n\,\partial x_1} & \frac{\partial^2 f}{\partial x_n\,\partial x_2} & \cdots & \frac{\partial^2 f}{\partial x_n^2} \end{bmatrix}$$

\subsection{Come risolvere un esercizio}

\subsection{Ricorda}
\subsection{Trucchetti}

\section{Funzioni implicite}
\subsection{Definizioni principali}

\subsection{Come risolvere un esercizio}

\subsection{Ricorda}
\subsection{Trucchetti}
\section{Integrali}
\subsection{Definizioni principali}

\subsection{Come risolvere un esercizio}

\subsection{Ricorda}
\subsection{Trucchetti}
\section{Altro}

\subsection{Nozioni utili}

\end{document}


